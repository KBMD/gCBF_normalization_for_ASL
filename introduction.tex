\section{Introduction}
Blood flow imaging of the brain has brought insights to neuroscience for over 50 years.\cite{Taber_2005} Computation of whole-brain, or global, cerebral blood flow (gCBF) from blood flow images has been important both when gCBF is an outcome measure of interest in its own right and when gCBF fluctuations create a nuisance that complicates identifying or interpreting relative changes in regional blood flow (rCBF). 

Early approaches for computing gCBF from positron emission tomography (PET) images included averaging over voxels remaining after a fixed threshold was applied as a percentage of peak rCBF; this was needed because of incomplete axial coverage of the brain from early scanners, and was reproducible because the images had low spatial resolution.[find a ref cited in the Perlmutter paper I shared with Stephanie] Another early approach used regional sampling of CBF images to estimate gCBF.[Perlmutter paper that I shared with Stephanie.] Eventually, technical advances in PET image resolution, structural imaging, computational power and cross-modal image registration allowed gCBF to be directly averaged across every voxel in the brain.

PET CBF images are made most commonly using the "autoradiographic" technique, i.e. by summing radioactivity over an interval of time after administration of the radiotracer. Equalizing the intensity of two such CBF images is most appropriately done by multiplicative scaling.[find a ref to go here from one of the citations in the Perlmutter paper, or maybe better yet invite Joel to join in and ask him to identify the most appropriate citation] 

By contrast, perfusion images created by the arterial spin labeling (ASL) MRI technique are created by subtracting two images, one in which arterial blood has been labeled using a radio frequency pulse ("tag"), and another control image without that label ("control"). Subtraction creates the possibility of a negative or positive additive bias across a CBF image, suggesting that additive rather than multiplicative correction may better equalize the intensity of two CBF images.

Here we test that hypothesis using ASL images acquired in \textit{check this date} 2006 as part of a pharmacological challenge MRI study in Parkinson disease (PD). Newer ASL implementations likely produce images of higher quality, and thus the magnitude of image intensity correction may be smaller. Still, greater need for correction may be ideal for the present purpose.