\section{Introduction}
Blood flow imaging of the brain has brought insights to neuroscience for over 50 years.\cite{Taber_2005} Computation of whole-brain, or global, cerebral blood flow (gCBF) from blood flow images is important both when gCBF is an outcome measure of interest in its own right and when gCBF fluctuations create a nuisance that complicates identifying or interpreting relative changes in regional blood flow (rCBF).\cite{Small_2004} 

Early approaches for computing gCBF from positron emission tomography (PET) images included averaging over voxels remaining after a fixed threshold was applied as a percentage of peak rCBF; this was needed because of incomplete axial coverage of the brain from early scanners, and was reproducible because the images had low spatial resolution.\cite{6609680} Another early approach used regional sampling of CBF images to estimate gCBF.\cite{6971299}\cite{Perlmutter_1985} Eventually, technical advances in PET image resolution, structural imaging, computational power and cross-modal image registration allowed gCBF to be directly averaged across every voxel in the brain.

PET CBF images are made most commonly using the ``autoradiographic'' technique, i.e. by summing radioactivity over an interval of time after administration of the radiotracer. These images are created by greater or lesser amounts of radiation flowing to different parts of the head, so (to the accuracy limits of the reconstruction algorithm) all voxel values are nonnegative, and voxels with higher positive numbers reflect greater CBF. When the arterial input function is measured, these images can be converted to a quantitative CBF map by scaling voxels multiplicatively based on their intensity. In fact, to a reasonable approximation, the scaling is constant across the image.[find a ref to go here from one of the citations in the Perlmutter paper, or maybe invite Joel to join in and ask him to identify the most appropriate citation] In many settings, changes in gCBF can be ignored by scaling all images to similar values, conventionally 50mL/hg/min. For our purposes, the key implication is that multiplicative scaling is an appropriate approach to equalizing the intensity of two autoradiographic CBF images. 

By contrast, perfusion images created by the arterial spin labeling (ASL) MRI technique are created by subtracting two images, one in which arterial blood flowing into the brain has been labeled using a spatially limited radio frequency pulse (``tag''), and a second image without that label (``control'').[Add a citation here to an early ref here for cASL and one for pASL, or a review of ASL methods.] Subtraction creates the possibility of a negative or positive additive bias across a CBF image, suggesting that additive rather than multiplicative correction may better equalize the intensity of two ASL CBF images.

Here we test that hypothesis using a set of ASL images acquired in \sout{2006?} as part of a pharmacological challenge MRI study in Parkinson disease (PD).\cite{Black_2010} Newer ASL implementations likely produce images of higher quality, but for the purpose of comparing correction methods, these older images may provide a more stringent test.