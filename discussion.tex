\section{Discussion}

Text Text Text Text Text Text  Text Text Text Text Text Text Text Text Text  Text Text Text Text Text Text Text Text Text Text.
Additional Requirements:
\subsection{Corrections}

Minor corrections to published articles can be communicated to the Frontiers Production Office at production.office@frontiersin.org. If you need to communicate important changes to an article please submit a General Commentary. Submit the article with the title “Erratum: Original Title of Article”.

\subsection{Commentaries on Articles}

At the beginning of your manuscript provide the citation of the article commented on.

\subsection{Focused Reviews}

For Tier 2 invited Focused Reviews the sections Introduction, Material and Methods, Results, and Discussion are recommended. In addition the authors must submit a short biography of the corresponding author(s). This short biography has a maximum of 600 characters, including spaces.

A picture (5 x 5 cm, in *.tif or *.jpg, min 300 dpi) must be submitted along with the biography in the manuscript and separately during figure upload.
Focused Reviews highlight and explain key concepts of your work. Please highlight a minimum of four and a maximum of ten key concepts in bold in your manuscript and provide the definitions/explanations at the end of your manuscript under “Key Concepts”. Each definition has a maximum of 400 characters, including spaces.

\subsection{Human Search and Animal Research}

All experiments on live vertebrates or higher invertebrates must be performed in accordance with relevant institutional and national guidelines and regulations. In the manuscript, authors must identify the committee approving the experiments and must confirm that all experiments conform to the relevant regulatory standards. For manuscripts reporting experiments on human subjects, authors must identify the committee approving the experiments and must also include a statement confirming that informed consent was obtained from all subjects. In Original Research Articles and Clinical Trial Articles these statements should appear in the Materials and Methods section.

\subsection{Clinical Trial Registration}

Clinical trials should be registered in a public trials registry in order to become the object of a publication at Frontiers. Trials must be registered at or before the start of patient enrollment. A clinical trial is defined as"any research study that prospectively assigns human participants or groups of humans to one or more health-related interventions to evaluate the effects on health outcomes."(\url{www.who.int/ictrp/en}). A list of acceptable registries can be found at \url{www.who.int/ictrp/en and www.icmje.org}.

\subsection{Inclusion of Proteomics Data}

Authors should provide relevant information relating to how the peptide/protein matches were undertaken, including methods used to process and analyze data, false discovery rates (FDR) for large-scale studies and threshold or cut-off rates for peptide and protein matches. Further information could include software used, mass spectrometer type, sequence database and version, number of sequences in database, processing methods, mass tolerances used for matching, variable/fixed modifications, allowable missed cleavages, etc.

Authors should provide as supplementary material information used to identify proteins and/or peptides. This should include information such as accession numbers, observed mass (m/z), charge, delta mass, matched mass, peptide/protein scores, peptide modification, miscleavages, peptide sequence, match rank, matched species (for cross species matching), number of peptide matches, ambiguous protein/peptide matches should be indicated, etc.
For quantitative proteomics analyses authors should provide information to justify the statistical significance including biological replicates, statistical methods, estimates of uncertainty and the methods used for calculating error.

For peptide matches with biologically relevant post-translational modifications (PTM) and for any protein match that has occurred using a single mass spectrum, authors should include this information as raw data, annotated spectra or submit data to an online repository (recommended option).
Authors are encouraged to submit raw or matched data and 2-DE images to public proteomics repositories. Submission codes and/or links to data should be provided within the manuscript.

\subsection{Data Sharing}

Frontiers supports the policy of data sharing, and authors are advised to make freely available any materials and information described in their article, and any data relevant to the article (while not compromising confidentiality in the context of human-subject research) that may be reasonably requested by others for the purpose of academic and non-commercial research. In regards to deposition of data and data sharing through databases, Frontiers urges authors to comply with the current best practices within their discipline.