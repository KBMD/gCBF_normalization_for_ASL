\section{Material \& Methods}

\subsection{Study participants}
Twenty-one nondemented, nondepressed adults age 40–75 with mild idiopathic PD (Hoehn and Yahr (1967) stages 1–3),\cite{6067254} treated with a stable dose of levodopa but no dopamine agonists, participated in the study. One subject was excluded for movement, as discussed below. Detailed inclusion and exclusion criteria are reported elsewhere.\cite{Black_2010}

\subsection{Subject behavior}
Each session included two perfusion MRI scans while the subject fixated a central crosshair surrounded by a circular checkerboard reversing at 8Hz and two control visual fixation scans with the crosshair but no checkerboard. \todo{Stephanie, check this next couple of sentences:} The data used here are drawn from a treatment study,\cite{Black_2010} but only from the scanning session on the placebo day before levodopa. 

\begin{table}
\caption{Maximum size of the Manuscript\label{Tab:01}}
{\begin{tabular}{lllll}
 & Abstract max. legth (incl. spaces) & Figures or tables & Manuscript max. length & Final PDF length\\\midrule
Clinical Case Study & & & &\\
Clinical Trial & & & &\\
Hypothesis and Theory & & & &\\
Methods & 2000 characters  & 15 & 12000 words & 12 pages\\
Original Research & & & &\\
Review & & & &\\
Technology Report & & & &\\
Focused Review & 2000 characters & 5 & 5000 words & 5 pages\\
\end{tabular}}{}
\end{table}

\begin{itemize}
\item Introduction: Succinct, with no subheadings.
\item Materials and Methods: This section may be divided by subheadings. This section should contain sufficient detail so that when read in conjunction with cited references, all procedures can be repeated.
\end{itemize}

