\section{Material \& Methods}

\subsection{Study participants}
Twenty-one nondemented, nondepressed adults age 40–75 with idiopathic PD (Hoehn and Yahr stages 1–3),\cite{6067254} treated with a stable dose of levodopa but no dopamine agonists, participated in the study. One subject was excluded for excessive head movement during the scan, as discussed below. Detailed inclusion and exclusion criteria are reported elsewhere.\cite{Black_2010} The study was approved by the Washington University Human Research Protection Office (IRB), and all subjects gave written informed consent prior to participation. 

\subsection{Subject behavior}
Each session included two perfusion MRI scans while the subject fixated a central crosshair surrounded by a circular checkerboard reversing at 8Hz and two control visual fixation scans with the crosshair but no checkerboard. The data used here are drawn from a treatment study,\cite{Black_2010} but only from the scanning session on the placebo day before levodopa. 

\subsection{MR image acquisition}
All MRI data were acquired on the Siemens 3.0T Tim Trio with matrix head coil. ASL images were acquired with the commercial Siemens pulsed arterial spin labeling (pASL) sequence.\cite{Wang_2003} The 15 transverse echo-planar readout slices had center-to-center slice distance 7.5 mm, 64$\times$64 voxels in plane with dimensions (3.4375mm)^2, repetition time (TR) 2.6 sec, echo time (TE) 13.0 msec, and flip angle 90°. An M_0 image was followed by 31 tag-control pairs for a total acquisition time for each ASL ``scan'' of 2.73 min.

Brain structure was assessed from sagittal MP-RAGE acquisitions with voxel size (1.0mm)^3, TR = 2.4 sec, TE = 3.08 msec, TI = 1000 msec, flip angle = 8°. The structural images for each subject were inspected visually, images of lower quality were rejected, and the remaining 1-4 MP-RAGE images for each subject were mutually registered.

\subsection{Defining volumes of interest}
\begin{itemize}
\item \textit{Whole-brain mask for each subject and their intersection for all subsequent analyses}
\item \textit{How GM & WM VOIs were defined from Freesurfer}
\item \textit{How the visual cortex region was defined for extracting the numbers that contributed to the Table.}
\end{itemize}

\subsection{Image registration and creation of CBF images}
Stephanie, go ahead and fill this subsection. Jon can help.

\subsection{Image scaling}
\subsubsection{Estimating modal CBF image intensity (``smooth mode'')}
\sout{Cite Avi if possible for why we chose mode not mean. Jon can add the method for finding the vertex of the parabola, which parabola, etc.} 
\subsubsection{Proportional (multiplicative) scaling}
\sout{This sub-sub-section and the next may be appropriate to combine.}
\subsubsection{Additive scaling}
\sout{Blah blah blah blah}

\subsection{Statistical analysis}
\sout{Blah blah blah blah}
