\section{Material \& Methods}

\subsection{Study participants}
Twenty-one nondemented, nondepressed adults age 40–75 with idiopathic PD (Hoehn and Yahr stages 1–3),\cite{6067254} treated with a stable dose of levodopa but no dopamine agonists, participated in the study. One subject was excluded for excessive head movement during the scan, as discussed below. Detailed inclusion and exclusion criteria are reported elsewhere.\cite{Black_2010} The data used here are drawn from a treatment study,\cite{Black_2010} but only from the scanning session on the placebo day before levodopa. The study was approved by the Washington University Human Research Protection Office (IRB), and all subjects gave written informed consent prior to participation.  

\subsection{Subject behavior}
Each session included two perfusion MRI scans while the subject fixated a central crosshair surrounded by a circular checkerboard reversing at 8Hz and two control visual fixation scans with the crosshair but no checkerboard. 

\subsection{MR image acquisition}
All MRI data were acquired on the Siemens 3.0T Tim Trio with matrix head coil. ASL images were acquired with the commercial Siemens pulsed arterial spin labeling (pASL) sequence.\cite{Wang_2003} The 15 transverse echo-planar readout slices had center-to-center slice distance 7.5 mm, 64$\times$64 voxels in plane with dimensions (3.4375mm)^2, repetition time (TR) 2.6 sec, echo time (TE) 13.0 msec, and flip angle 90°. An M_0 image was followed by 31 tag-control pairs for a total acquisition time for each ASL ``scan'' of 2.73 min.

Brain structure was assessed from sagittal MP-RAGE acquisitions with voxel size (1.0mm)^3, TR = 2.4 sec, TE = 3.08 msec, TI = 1000 msec, flip angle = 8°. The structural images for each subject were inspected visually, images of lower quality were rejected, and the remaining 1-4 MP-RAGE images for each subject were mutually registered.


\subsection{Image registration and creation of CBF images}
The 63 frames of the ASL run were rigidly aligned using a validated method (Black et al., 2001) and summed to facilitate later alignment steps. Each frame was smoothed using a Gaussian filter with kernel measuring 7.35 mm (FWHM), and cerebral blood flow (CBF) was computed in each voxel for each tag-control pair (Wang et al., 2003). The summed within-run-aligned EPI images were mutually aligned within each subject and summed, and the resulting image was affine registered to a target image in Talairach and Tournoux space made using validated methods from the structural MR images from these subjects (Black et al., 2004). From this registration, the 31 tag-control-pair CBF images were resampled into atlas space using matrix multiplication to avoid multiple resampling steps, and averaged to create a single atlas-registered CBF image for each ASL run. 
Performed motion censoring as per Power et al. One subject was excluded after motion data indicated that there was motion artifact in over 50? percent of frames.

\subsection{Image scaling}
\subsubsection{Estimating modal CBF image intensity (``smooth mode'')}
\sout{Cite Avi if possible for why we chose mode not mean. Jon can add the method for finding the vertex of the parabola, which parabola, etc.} 
\subsubsection{Proportional (multiplicative) scaling}
\sout{This sub-sub-section and the next may be appropriate to combine.}
\subsubsection{Additive scaling}
\sout{Blah blah blah blah}

\subsection{Statistical analysis}
To determine activation analysis, we used a two-level, mixed effects approach.  First, for each study subject, changes in rCBF were identified using SPM__ software and a voxelwise general linear model that included task (checkerboard) vs. control (crosshair fixation). A statistical parametic image for each subject was generated from the task contrast and were used as input data for a second-level SPM analysis for a voxelwise general linear model that includined a covariate for subject age and sex. Statistical inference was performed at each voxel with a one-sample t test (ie testing across subjects whether the task contrast images are significantly less than or greater than zero). 
Multiple-comparisons correction was performed at the cluster level with a false discovery rate set at p = 0.05. 

Approximate anatomical locations were provided by the Talairach Daemon client (www.talairach.org) (Lancaster el at 2000). 
To determine the effect of correcting for the gCBF on regional analyses, all statistical analyses were performed in triplicate- first, before removing the gCBF effect (unnormalized) and then repeated after removing the gCBF effect by scaling all input images multiplicatively (multiplied to normalize) and proportionally (shifted to normalize), so that the global mean CBF for every image was 


\subsection{Defining volumes of interest}
\begin{itemize}
\item \textit{Whole-brain mask for each subject and their intersection for all subsequent analyses}
\item \textit{How GM & WM VOIs were defined from Freesurfer}
\item \textit{How the visual cortex region was defined for extracting the numbers that contributed to the Table.}
\end{itemize}






