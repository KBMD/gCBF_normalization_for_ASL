\section{Material \& Methods}

\subsection{Study participants}
Twenty-one nondemented, nondepressed, ambulatory adults age 40–75 with idiopathic PD, treated with a stable dose of levodopa but no dopamine agonists, participated in the study. One subject was excluded for excessive head movement during the scan, as discussed below. Detailed inclusion and exclusion criteria were reported previously.\cite{SYN115_2010_AAN_RCT} Subjects were enrolled in a Phase 2a dose-finding study,\cite{Black_2010} but here we use only data acquired on the placebo day when subjects were in the ``practical off state'' (\textit{i.e.} no antiparkinsonian medications for at least 9 hours). The study was approved by the Washington University Human Research Protection Office (IRB), and all subjects provided written documentation of informed consent prior to participation.

\subsection{Subject behavior}
Each scanning session included two perfusion MRI scans while the subject fixated a central crosshair surrounded by a circular checkerboard reversing at 8Hz and two control visual fixation scans with the crosshair but no checkerboard. 

\subsection{MR image acquisition}
All MRI data were acquired on the Siemens 3T Tim Trio with matrix head coil. ASL images were acquired with the commercial Siemens pulsed arterial spin labeling (pASL) sequence.\cite{Wang_2003} Fifteen transverse echo-planar readout slices were acquired with center-to-center slice distance 7.5 mm, 64$\times$64 voxels in plane with dimensions (3.4375mm)\textsuperscript{2}, repetition time (TR) 2.6 sec, echo time (TE) 13.0 msec, and flip angle $90\textdegree.$ An M\textsubscript{0} image was followed by 31 tag-control pairs for a total acquisition time for each ASL ``scan'' of 2.73 min.

Brain structure was assessed from sagittal MP-RAGE acquisitions with voxel size (1.0mm)\textsuperscript{3}, TR = 2.4 sec, TE = 3.08 msec, TI = 1000 msec, flip angle = 8°. The structural images for each subject were inspected visually, images of lower quality were rejected, and the remaining 1-4 MP-RAGE images for each subject were mutually registered.

\subsection{Image registration and creation of CBF images}
The 63 frames of the ASL run were rigidly aligned using a validated method\cite{Black_2001} and summed to facilitate later alignment steps. Each frame was smoothed using a Gaussian filter with kernel measuring \sout{7.35 mm} (FWHM), and cerebral blood flow (CBF) was computed in each voxel for each tag-control pair as described.\cite{Wang_2003} The summed, aligned EPI images from each run were mutually aligned within each subject and summed across runs, and the resulting image was affine registered to a target image in Talairach and Tournoux space made using validated methods from the structural MR images from these subjects.\cite{15130735} The registration matrix from this step and the matrices from the within-run mutual registration step were used to resample the 31 tag-control pair CBF images from each run into atlas space using matrix multiplication with a single resampling step. To minimize motion-related artifact we removed tag--control pairs if framewise displacement in either EPI image exceeded 0.9mm.\cite{23861343} The remaining CBF images in atlas space were averaged to create a single atlas-registered CBF image for each ASL run. One subject was excluded from further analysis because over half of his frame pairs were removed due to head motion.

\subsection{Image intensity correction}
\subsubsection{Estimating modal CBF image intensity}
The image intensity histograms were constructed with bins 1 unit wide, so were not smooth. We chose to normalize image intensity based on the idealized peak of this distribution (which if there were no noise would be the mode, \textit{i.e.} the most common voxel intensity in the image).\cite{Ojemann_1997} Specifically, the method of least squares was used to identify the vertex of the parabola that best fit the histogram \sout{describe here which part of the histogram we fit the data to---I think it was from the lowest bin whose frequency was above a certain percent threshold of the true mode to the highest bin above that same threshold}.

\verb|Jon, please correct that last sentence.|
\subsubsection{Additive and multiplicative intensity correction}
Each input image was corrected in two ways: multiplicatively (multiplying every voxel in the image by $50/mode$), and additively (adding $50-mode$ to every voxel), so that the modal CBF for every corrected image was 50 (nominal units mL/hg/min).

\subsection{Defining volumes of interest}
Gray matter (GM), white matter (WM) and visual cortex volumes of interest (VOIs) were defined from each subject's MP-RAGE image by \href{https://surfer.nmr.mgh.harvard.edu/}{FreeSurfer}.\cite{Desikan2006968} VOIs were limited to voxels that were represented in every image in every subject; this step excluded much of the inferior occipital cortex in the visual cortex VOI.\cite{Black_2010} 

\subsection{Statistical analysis}
To determine the effect of gCBF normalization on task effect, all statistical analyses were performed in triplicate, one for each set of images: uncorrected (before removing the gCBF effect), multiplicatively normalized, and additively corrected.

\subsubsection{VOI analysis}
Mean CBF across all voxels in a VOI was computed from each atlas-registered CBF image. The ratio of GM VOI mean to WM VOI mean was computed for each CBF image and compared across image sets using a repeated measures analysis of variance (ANOVA). Activation with visual stimulation in a VOI was tested for statistical significance in each image set using \sout{a repeated measures ANOVA}. 

\subsubsection{Statistical images}
To identify regions activated by the visual task, we used a two-level, mixed effects approach.  First, for each study subject, changes in rCBF were identified using SPM12b software and a voxelwise general linear model that included task (checkerboard) vs. control (crosshair fixation). A t image for each subject was generated from the task contrast. These t images were used as input for a second-level SPM analysis using a voxelwise general linear model that included a covariate for subject age and a factor for sex. Statistical inference was performed at each voxel with a one-sample t test (\textit{i.e.} testing whether the task contrast images are significantly less than or greater than zero, across subjects). Multiple comparisons correction was performed with the cluster false discovery rate set at p = 0.05. 

Approximate anatomical locations of peaks in the statistical images were provided by the \href{http://www.talairach.org}{Talairach Daemon client}.\cite{20408222}

\verb|Jon, see comments on this section.|
