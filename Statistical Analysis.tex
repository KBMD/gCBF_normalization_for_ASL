To determine activation analysis, we used a two-level, mixed effects approach.  First, for each study subject, changes in rCBF were identified using SPM__ software (ww.fil.ion.ucl.ac.uk/spm/) and a voxelwise general linear model that included task (checkerboard) vs. control (crosshair fixation). A statistical parametic image for each subject was generated from the task contrast and were used as input data for a second-level SPM analysis for a voxelwise general linear model that includined a covariate for subject age and sex. Statistical inference was performed at each voxel with a one-sample t test (ie testing across subjects whether the task contrast images are significantly less than or greater than zero). Multiple-comparisons correction was performed at the cluster level with a false discovery rate set at p = 0.05. 

Approximate anatomical locations were provided by the Talairach Daemon client (www.talairach.org) (Lancaster el at 2000). 
To determine the effect of correcting for the gCBF on regional analyses, all statistical analyses were performed in triplicate- first, before removing the gCBF effect (unnormalized) and then repeated after removing the gCBF effect by scaling all input images multiplicatively (multiplied to normalize) and proportionally (shifted to normalize), so that the global mean CBF for every image was 50 ml hg−1 min−1. 
