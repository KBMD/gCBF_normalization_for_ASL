\section{Results}
We assessed the effect of additive or multiplicative intensity equalization in several ways: by examining the voxel intensity distribution (by frequency histogram), by judging image quality visually and by quantitative GM:WM CBF ratios, and by the images' ability to detect the effect of visual stimulation in a partial visual cortex VOI. 

\subsection{Voxel intensity distribution (histogram)}
The original CBF images contained a reasonable distribution of voxel intensities except that they appeared shifted leftwards to varying degrees, so that many voxels in the brain had physiologically implausible negative values (the 3 curves in the histogram in the \verb|first| row of Fig. \ref{fig:BlueGreenRed} reflect three ASL images from the same subject; a transverse section from each of these images is shown to the right of the histogram). Multiplicative normalization of course produced an image with an equal fraction of negative voxel values, though the normalized image's mode was now 50 (histogram in \verb|second| row of Fig. \ref{fig:BlueGreenRed}). Additive normalization produced a voxel intensity distribution that reflects the physiological expectation (histogram in \verb|third| row of Fig. \ref{fig:BlueGreenRed}). 

\subsection{Image quality}
Images normalized additively (\sout{first? second? third?} row in Fig. \ref{fig:BlueGreenRed}) present much clearer gray:white contrast on visual inspection. We tested this impression quantitatively using the ratio in each subject of the mean GM CBF to the mean WM CBF. The mean $\pm$ SD ratio for the original CBF images was MMM $\pm$ SSS. For the multiplicatively normalized images it was MMM $\pm$ SSS, and for the additively normalized images it was MMM $\pm$ SSS. These differed significantly ($p = .XXX$), with post hoc tests significant for additive vs. original ($p = .XXX$), multiplicative vs. original ($p = .XXX$)\verb|, and additive vs. multiplicative ($p = .XXX$)|.

\subsection{Task activation: \textit{a priori} volume of interest}
  \begin{itemize} 
    \item mean change (hard to judge) and 
    \item t-test (shift50 is slightly better than mult50 or unshifted)
  \end{itemize}

\subsection{Task activation: statistical image}
\sout{By SPM (shift50 is much better than unshifted, but it's debatable which of the shifts is better)}

